% Chapter 7

\chapter{CONCLUSIONS} % Write in your own chapter title

\section{SUMMARY}
This is a classification system which classifies any email as ham or spam. The advantage of this system is that classification is done on the basis of contextual similarity. The system uses a Snowball stemmer to stem words and a TF-IDF Vectorizer to obtain a term-by-document frequency matrix. Singular Value Decomposition (SVD) is carried out to reduce the dimensionality following which the set of ‘k’ (k can be varied) largest (strongest) features are selected. Finally, the selected features are fitted to a neural network which classifies each input email as spam or ham. In the Corpus Based Thesaurus approach, we find the cosine similarity between every two elements in the TF-IDF matrix. However, owing to severe memory and performance restrictions, this is difficult to obtain completely. The results of performance evaluation are very encouraging and show promising values for precision, recall and F measure of 0.846, 0.842 and 0.844 respectively. While looking at domain specific classification, precision, recall and F-measure scores obtained are all above 0.8.

\section{CRITICISMS}
The errors in the stemming propagate down to all the modules of the system and may influence final output. Neural networks typically have slow training speed. They are also very likely to get trapped in a local minimum, thereby rendering the given task incomplete. The Corpus Based Thesaurus approach is very memory intensive and requires large amount of processing capability to complete entirely. More often than not, the CBT approach typically results in a hardware crash. Finally, some test cases resulted in false positive and rejection of legitimate mail. Most users would rather receive the spam email than lose the legitimate emails.

\section{FUTURE WORK}
Feature representation methods and classification algorithms are two critical components for spam filtering and even for data mining. This project investigates the effectiveness of semantic similarity measure feature representation methods and neural network algorithms, the combination of these two methods achieved very promising results for spam filtering. However, using improved neural network approaches such as Adaptive Backpropagation (ABPNN) are likely to improve classification efficiency even further, by eliminating the traditional drawbacks of neural networks. The ABPNN approach is likely to involve tuning the learning rate at every iteration. Our future work is to investigate more efficient feature representation methods and classification algorithms for data mining and apply the method to other data mining applications.
